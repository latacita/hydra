%%==================================================================%%
%% Author : Tejedo Gonz�lez, Daniel                                 %%
%%          S�nchez Barreiro, Pablo                                 %%
%% Version: 1.0, 10/12/2012                                         %%                   
%% Version: 2.0, 11/03/2013                                         %%                   
%%                                                                  %%
%% Memoria del Proyecto Fin de Carrera                              %%
%% semantica, introduccion                                             %%
%%==================================================================%%

Tal y como indica la metodolog�a impuesta por la t�cnica de Ingenier�a de Lenguajes Dirigida por Modelos, la �ltima parte del desarrollo de nuestro lenguaje consiste en la implementaci�n la sem�ntica. La sem�ntica es la responsable de la ejecuci�n de los programas escritos en el lenguaje en cuesti�n. En nuestro caso, la ejecuci�n pasa por validar las restricciones que hayamos escrito en la configuraci�n deseada. Adem�s, en este cap�tulo tambi�n hablaremos brevemente del despliegue de la herramienta

Para tratar todos estos temas, el cap�tulo se estructurar� del siguiente modo: en primer lugar se hablar� de los mecanismos utilizados para llevar a cabo la implementaci�n de la sem�ntica.  En siguiente lugar, se ense�ar�n las pruebas a las que la sem�ntica ha sido sometida, mucho m�s exhaustivas y determinantes que las anteriores, ya que engloban el total funcionamiento de la aplicaci�n. A continuaci�n habr� una secci�n sobre la creaci�n de la interfaz del editor y su funcionamiento, adem�s de su integraci�n con Hydra. Para finalizar, se incluir� una breve secci�n acerca del despliegue de la herramienta.

