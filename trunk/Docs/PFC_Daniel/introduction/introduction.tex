%%==================================================================%%
%% Author : Tejedo Gonz�lez, Daniel                                 %%
%%          S�nchez Barreiro, Pablo                                 %%
%% Version: 1.0, 14/11/2012                                         %%                   %%                                                                  %%
%% Memoria del Proyecto Fin de Carrera                              %%
%% Introducci�n, archivo ra�z                                       %%
%%==================================================================%%

\chapterheader{Introducci�n}{Introducci�n}
\label{chap:introduction}

%%==================================================================%%
%% NOTA(Pablo): Ponle nombre a tu proyecto, com m�dulo dentro de    %%
%%              de Hydra                                            %%
%%              Comienza nombrando a tu proyecto, no a Hydra. Tu    %%
%%              proyecto es el producto principal, Hydra es el      %%
%%              producto marginal                                   %%
%%              Tampoco menciones el AMPLE, porque en este caso     %%
%%              no es necesario, ya que no has estado trabajando    %%
%%              para dicho proyecto                                 %%
%%                                                                  %%     %%==================================================================%%

Esta memoria de Proyecto Fin de Carrera presenta un entorno para la especificaci�n y validaci�n de restricciones en �rboles de caracter�sticas con cardinalidad. El entorno ha sido desarrollado dentro del contexto de la aplicaci�n Hydra, una herramienta para el modelado de caracter�sticas. Este cap�tulo introduce la situaci�n anterior del proyecto Hydra y los conceptos implicados en el mismo, as� como los nuevos conceptos que esta ampliaci�n incorpora. Se describir�n tambi�n los objetivos que este trabajo pretende alcanzar y la motivaci�n que subyace tras ellos. Por �ltimo, se indicar� la estructura que presentar� este documento.


\chaptertoc

\section{Introducci�n}
\label{sec:intr:introduction}

%%==================================================================%%
%% Author : Tejedo Gonz�lez, Daniel                                 %%
%%          S�nchez Barreiro, Pablo                                 %%
%% Version: 1.0, 14/11/2012                                         %%
%% Version: 1.0, 21/01/2013                                         %%
%%
%% Memoria del Proyecto Fin de Carrera                              %%
%% Introducci�n/Introducci�n                                        %% %%==================================================================%%

El principal objetivo de este Proyecto de Fin de Carrera es extender la herramienta \emph{Hydra}~\cite{} para que soporte la especificaci�n y validaci�n de restricciones que contengan caracter�sticas con cardinalidad. Dicho objetivo resultar�, como es l�gico, confuso para el lector no familiarizado con las l�neas de productos software~\cite{pohl:2005, kakola:2006} en general, y con los �rboles de caracter�sticas con cardinalidad~\cite{czarnecky:2005}, en particular. Por tanto, intentaremos introducir de forma breve al lector en estos conceptos.

%%=================================================================%%
%% NOTA(Pablo) : Demasiado complejo y poco claro
%%=================================================================%%
%%
%% continuar el desarrollo de la herramienta Hydra all� d�nde
%% se dej�. Pero para entender un poco mejor las caracter�sticas
%% del entorno que aqu� se ha desarrollado conviene explicar un
%% poco las razones que motivaron la creaci�n del proyecto Hydra
%% en primera instancia. Desde su nacimiento, Hydra ha pretendido
%% convertirse en la aplicaci�n m�s completa para trabajar con
%% L�neas de Producto Software, dado que no existe ninguna que
%% ofrezca una serie de caracter�sticas de manera conjunta.
%% Trabajar con L�neas de productos software conlleva a su vez
%% trabajar con una nada desde�able cantidad de conceptos
%% �ntimamente vinculados a ellas. En los pr�ximos p�rrafos se
%% describir� lo que es una l�nea de productos software y los
%% conceptos subyacentes que nos permiten trabajar con ellas.
%%
%%=================================================================%%

El objetivo de una \emph{L�nea de Productos Software}~\cite{clements:2002} es crear la infraestructura adecuada para una r�pida y f�cil producci�n de sistemas software similares, destinados a un mismo segmento de mercado. Las l�neas de productos software se pueden ver como an�logas a las l�neas de producci�n industriales, donde productos similares o id�nticos se ensamblan y configuran a partir de piezas prefabricadas bien definidas. Un ejemplo cl�sico de l�nea de producci�n industrial es la fabricaci�n de autom�viles, donde se pueden crear decenas de variaciones de un �nico modelo de coche con un solo grupo de piezas cuidadosamente dise�adas mediante una l�nea de montaje espec�ficamente dise�ada para configurar y ensamblar dichas piezas.

Ya dentro del mundo del software, el desarrollo de software, por ejemplo, para tel�fonos m�viles implica la creaci�n de productos con caracter�sticas muy parecidas, pero diferenciados entre ellos. Por ejemplo, una aplicaci�n de agenda personal podr� ofrecer diferentes funcionalidades en funci�n de si el terminal m�vil posee GPS (\emph{Global Positioning System}), acceso a mapas o \emph{bluetooh}. Por tanto, el objetivo de una l�nea de productos software es crear una especie de l�nea de montaje donde una aplicaci�n de agenda personal como la mencionada se pueda construir de la forma m�s eficiente posible de acuerdo a las caracter�sticas concretas de cada terminal espec�fico.

Para construir una l�nea productos software, el primer paso es analizar qu� caracter�sticas comunes y variables poseen cada uno de los productos que tratamos de producir. Para realizar dicho an�lisis de la variabilidad de una familia de productos se utilizan diversas t�cnicas. Las m�s utilizadas actualmente son la creaci�n de �rboles de caracter�sticas~\cite{kang:1990, czarnecky:2005, danilo:2003} y los lenguajes espec�ficos de dominio~\cite{martin:2010}. En este proyecto nos centraremos en la primera opci�n.

Un �rbol de caracter�sticas~\cite{} es un tipo de modelo que especifica, tal como su nombre indica, en forma de �rbol las caracter�sticas que puede poseer un producto concreto perteneciente a una familia de productos, indicando qu� caracter�sticas son comunes a todos los productos y cu�les son variables, as� como las razones por las cuales son variables.

Por ejemplo, en una l�nea de productos de agendas personales para tel�fonos m�viles, toda agenda personal debe permitir anotar eventos a los que debemos asistir en un futuro cercano. Por tanto, esta caracter�stica ser�a una caracter�stica obligatoria para todas las agendas personales. Sin embargo, ciertas agendas, dependiendo del precio que el usuario final est� dispuesto a pagar y las caracter�sticas t�cnicas de cada terminal, podr�an ofrecer la funci�n de geolocalizar el lugar del evento al que debemos asistir, y calcular la ruta �ptima desde el lugar que le indiquemos a dicho lugar de destino. Esta �ltima caracter�stica ser�a opcional, y podr�a no estar incluida en ciertas agendas personales instalados en terminales concretos.

Para obtener un producto espec�fico dentro de una l�nea de productos software, el cliente debe especificar qu� caracter�sticas concretas desea que posea el producto que va a adquirir. Es decir, en t�rminos t�cnicos, debe crear una \emph{configuraci�n} del �rbol de caracter�sticas. Obviamente, no toda selecci�n de caracter�sticas da lugar a una configuraci�n v�lida. Por ejemplo, toda configuraci�n debe contener al menos el conjunto de caracter�sticas que son obligatorias para todos los productos. De igual forma, puede ser obligatorio escoger al menos una caracter�sticas de entre una serie de alternativas. Por ejemplo, la agenda personal podr�a estar disponible en castellano, ingl�s y franc�s. En este caso ser�a posible seleccionar cualquiera de las tres alternativas, pero al menos una deber�a incluirse en nuestro producto. Tambi�n ser�a posible indicar que podemos seleccionar un �nico idioma, es decir, que no podemos instalar una agenda personal que soporte de forma simult�nea dos idiomas distintos.

La mayor�a de estas restricciones se pueden especificar usando la sintaxis propia de los �rboles de caracter�sticas. No obstante, existen una serie de restricciones que no se pueden modelar con la sintaxis propia de los �rboles de caracter�sticas. Un ejemplo de tal tipo de restricci�n son las relaciones de dependencias entre caracter�sticas. Por ejemplo, la selecci�n de una caracter�stica de c�lculo de rutas �ptimas podr�a necesitar para funcionar que  estuviesen instalados los servicios de mapas y geolocalizaci�n. Dichas tres caracter�sticas podr�an no aparecer relacionadas en el �rbol de caracter�sticas, por lo que tendr�amos que definir dicha restricci�n como una restricci�n externa.

Estas restricciones externas se suelen especificar utilizando f�rmulas de l�gica proposicional~\cite{kleine:1999}. Los �tomos de dichas f�rmulas son las caracter�sticas del sistema. Dichos �tomos se eval�an a verdadero si las caracter�sticas correspondientes est�n seleccionadas, y a falso en caso contrario. Por ejemplo, la restricci�n anteriormente expuesta podr�a especificarse como $CalculoRutasOptimas \Rightarrow (Mapas \wedge Geolocalizacion)$.

Para que estas restricciones sean de utilidad, adem�s de especificarlas, debemos comprobar que se satisfacen para las diferentes configuraciones creadas. En los �ltimos a�os se han ido creando diversas t�cnicas y herramientas para el an�lisis y validaci�n de dichas restricciones~\cite{}.

Paralelamente al problema de la especificaci�n y validaci�n de las restricciones externas, se han ido incorporando diversas modificaciones y novedades a los modelos de �rboles de caracter�sticas en los �ltimos a�os. Por ejemplo, se han introducido conceptos como las \emph{referencias entre caracter�sticas}~\cite{czarnecky:2005} y \emph{atributos}~\cite{} para las caracter�sticas. Uno de estos conceptos, simple pero importante, ha sido el de caracter�stica clonable~\cite{batory:2005, czarnecky:2005}. Una caracter�stica clonable es una caracter�stica que puede aparecer un n�mero variable de veces dentro de un producto.

Por ejemplo, supongamos que tenemos una red de sensores para la monitorizaci�n y regulaci�n del nivel de humedad de un determinado recinto, por ejemplo, de un invernadero. Dependiendo de donde fu�semos a instalar dicha red, podr�amos necesitar un n�mero diferente de sensores. Adem�s, dependiendo de donde instal�semos cada sensor, podr�amos configurar cada sensor de forma diferente. Por ejemplo, ciertos sensores podr�an necesitar tener capacidades de enrutamiento, ser tolerantes a fallo o poseer modos de hibernaci�n para disminuir el consumo de energ�a. Por tanto, en dicho sistema ser�a interesante modelar \emph{Sensor} como una caracter�stica que se puede clonar, es decir, crear un n�mero variable de instancias de la misma, y donde cada clon fuese a su vez configurable con ciertas caracter�sticas.

La incorporaci�n de las caracter�sticas clonables a los �rboles de caracter�sticas hace que los mecanismos utilizados hasta ahora para especificar y evaluar restricciones externas hayan quedado obsoletos. Dado que las caracter�sticas clonables no se seleccionan sino que se clonan, ya no podemos evaluar una caracter�stica clonable a verdadero o falso dependiendo de si est� o no seleccionada. El concepto de \emph{estar seleccionada} desaparece en el caso de las caracter�sticas clonables.

Para solventar dicho problema, el profesor Pablo S�nchez, dentro del Departamento de Matem�ticas, Estad�stica y Computaci�n, ha desarrollado un nuevo lenguaje para la especificaci�n y validaci�n de restricciones externas a los �rboles de caracter�sticas donde dichas restricciones pueden contener \emph{caracter�sticas clonables}. Dicho lenguaje se denomina \emph{HCL (Hydra Constraint Language}).

El objetivo de este Proyecto Fin de Carrera es implementar un editor que permita especificar y validar restricciones especificadas en HCL, es decir, restricciones sobre �rboles de caracter�sticas que puedan incluir caracter�sticas clonables. Dicho editor se debe integrar en la herramienta para el modelado y configuraci�n de �rboles de caracter�sticas denominada \emph{Hydra}, desarrollada tambi�n por el profesor Pablo S�nchez, en colaboraci�n con un antiguo alumno suyo de la Universidad de M�laga, Jos� Ram�n Salazar. Con esto esperamos haber aclarado el primer p�rrafo de esta secci�n al lector no familiarizado con las l�neas de productos software y/o los �rboles de caracter�sticas.

Hydra se distribuye actualmente como un plugin para Eclipse, y ha sido desarrollada utilizando modernas t�cnicas de \emph{Ingenier�a de Lenguajes Dirigida por Modelos}~\cite{anneke:2008}. Dichas t�cnicas permiten una r�pida y c�moda creaci�n de entornos de edici�n y evaluaci�n de lenguajes tanto visuales como textuales mediante la especificaci�n de una serie de elementos b�sicos a partir de los cuales se genera una gran cantidad de artefactos, reduciendo los tiempos de desarrollo y costo asociado al desarrollo de dichos entornos. El editor desarrollado en este Proyecto Fin de Carrera deber� distribuirse tambi�n como un plugin para Eclipse, instalable sobre \emph{Hydra}. Para su desarrollo se usar� tambi�n un enfoque de \emph{Ingenier�a de Lenguajes Dirigida por Modelos}~\cite{anneke:2008}.

Tras esta introducci�n, el resto del presente cap�tulo se estructura como sigue: La Secci�n~\ref{sec:intr:sle} proporciona unas nociones b�sicas sobre la \emph{Ingenier�a de Lenguajes Dirigida por Modelos}, nociones que son necesarias para poder entender la planificaci�n del presente proyecto, la cual se describe en la Secci�n~\ref{sec:intr:planning}. Por �ltimo, la Secci�n~\ref{sec:intr:estructura} describe la estructura general del presente documento.

%%===========================================================================================
%% NOTA(Pablo): Esto es demasiado detalle para una introducci�n. Aparte no se entiende nada
%%===========================================================================================
%%  Para crear el lenguaje de dominio espec�fico que permita la especificaci�n de
%%  restricciones externas entre caracter�sticas se ha utilizado la t�cnica de
%%  Ingenier�a de Lenguajes Dirigido Por Modelos. Esta t�cnica esa una aproximaci�n de
%%  la Ingenier�a Dirigida Por Modelos desde el punto de vista de la Teor�a de Lenguajes
%%  Formales. La Ingenier�a Dirigida Por Modelos es una metodolog�a de desarrollo de
%%  software que se basa en la construcci�n de la aplicaci�n final a partir de uno o m�s
%%  modelos abstractos que representen el comportamiento y la funcionalidad de la misma.
%%  Mediante la modificaci�n de los distintos par�metros configurables dentro de los
%%  modelos ser� posible la construcci�n de herramientas diversas para un problema
%%  espec�fico de manera relativamente sencilla, bastando simplemente con crear una
%%  serie de instancias v�lidas de los modelos representativos de nuestra aplicaci�n.
%%
%%  La Ingenier�a Dirigida Por Modelos se puede usar para crear nuevos lenguajes de
%%  programaci�n, especialmente DSLs o Lenguajes Espec�ficos de Dominio. Basta con
%%  imaginar uno o varios metamodelos cuyas instancias v�lidas representen una l�nea
%%  o estructura correcta de c�digo. Estos metamodelos forman la llamada sintaxis
%%  abstracta de nuestro lenguaje, pues representan de manera abstracta todas las
%%  posibles representaciones gr�ficas o textuales que podemos hacer dentro de ese
%%  lenguaje.
%%
%% A partir del metamodelo que representa la sintaxis abstracta podremos construir
%% una serie de modelos que sean instancias del mismo. Esto se denomina la sintaxis
%% concreta, es decir, la representaci�n concreta de una de las m�ltiples posibilidades
%% de instanciaci�n del modelo abstracto. De todos modos, si queremos que las palabras
%% admitidas por nuestro lenguaje sean expresadas de otro modo aparte de mediante modelos,
%% no nos veremos liberados de la tarea de tener que expresar una gram�tica formal con
%% producciones, pero hace que la construcci�n de la misma est� acotada dentro de unos
%% t�rminos delimitados por el modelo constru�do, lo cual favorece la sencillez de la
%% gram�tica y su comprensi�n. Esta gram�tica servir� para identificar si la creaci�n de
%% expresi�n concreta a la que puede ser transformada es viable mediante el metamodelo
%% abstracto.
%%
%% Una vez hemos construido los medios necesarios para comprobar que las expresiones que
%% fabriquemos son correctas, necesitamos idear el modo de que las �rdenes que esas l�neas
%% producen sean ejecutadas. Para ello entra en juego la sem�ntica del lenguaje, es decir,
%% la encargada de aportar un significado real a todas las expresiones que hayamos
%% construido. Dicho de otro modo, la sem�ntica es la encargada de implementar las
%% funciones derivadas de las �rdenes descritas por cada una de las sintaxis concretas
%% posibles. Poniendo un ejemplo, el metamodelo que represente la sintaxis abstracta de
%% java puede generar una infinidad de sintaxis concretas, entre ellas
%% System.out.println("Hola Mundo"). Y para que esa instrucci�n escriba el mensaje
%% Hola Mundo por pantalla es necesaria una sem�ntica que as� lo indique. Para
%% implementar la sem�ntica del lenguaje no existe otro m�todo que la programaci�n
%% directa en un lenguaje determinado, en nuestro caso Java.

%%===========================================================================================
%% NOTA(Pablo): Esto es a estas alturas no se entiende. Es m�s propio de un sumario para el
%%              final
%%===========================================================================================
%%
%% Por �ltimo y para terminar esta introducci�n, conviene contextualizar un poco el
%% trabajo que hemos hecho mediante un ejemplo concreto de lo que se quiere implementar.
%% Nuestro editor para la especificaci�n y validaci�n de restricciones en �rboles de
%% caracter�sticas con caracter�sticas clonables tiene que implementar la siguiente
%% funcionalidad: \\
%%
%% 1 - Obligar a que todas los ficheros de restricciones empiecen con una l�nea de import
%%     que servir� para importar el modelo de caracter�sticas sobre el cual se analizar�n
%%     las restricciones.
%% 2 - Escribir restricciones v�lidas para ese modelo (ejemplo: "(a or b) implies (c and d);", %%     m�s adelante se hablar� en detalle del lenguaje y de las operaciones que implementa).
%% 3 - Detectar que las caracter�sticas a las que estamos aplicando esas restricciones en
%%     efecto se hallan en el modelo que ha sido importado.
%% 4 - Cargar una instancia de ese modelo (lo que llamamos modelo de especializaci�n) y mirar
%%     si para ella las restricciones que han sido especificadas se cumplen.
%%
%%===========================================================================================


\section{Objetivos}
\label{sec:intr:motivation}

%%==================================================================%%
%% Author : Tejedo Gonz�lez, Daniel                                 %%
%%          S�nchez Barreiro, Pablo                                 %%
%% Version: 1.0, 14/11/2012                                         %%                   %%                                                                  %%
%% Memoria del Proyecto Fin de Carrera                              %%
%% Introducci�n/Introducci�n                                        %%
%%==================================================================%%

Como ya se ha comentado en la secci�n de introducci�n, no existe ninguna herramienta que posea de forma conjunta una serie de elementos de inter�s para el modelado de L�neas de Productos Software y �rboles de Caracter�sticas. M�s concretamente, no existe ninguna herramienta que contemple el modelado, configuraci�n y validaci�n de caracter�sticas clonables. Estas caracter�sticas son imprescindibles para el modelado de la variabilidad estructural. Por lo tanto, el objetivo de Hydra siempre fue suplir esas carencias, en la medida de lo posible.

Concretando m�s en concreto, los objetivos de Hydra se pueden clasificar en los 4 que se enumeran a continuaci�n: \\

1. Desarrollar un editor completamente gr�fico y amigable al usuario para la construcci�n de modelos de caracter�sticas, incluyendo soporte para el modelado de caracter�sticas clonables.

2. Desarrollar un editor textual y una sintaxis propia para la especificaci�n de restricciones entre caracter�sticas, incluyendo restricciones que involucren caracter�sticas clonables.

3. Desarrollar Un editor gr�fico, asistido y amigable al usuario para la creaci�n de configuraciones de modelos de caracter�sticas, incluyendo soporte para la configuraci�n de caracter�sticas clonables.

4. Crear un validador que compruebe que las configuraciones creadas satisfacen las restricciones definidas para el modelo de caracter�sticas, incluso cuando estas restricciones contengan caracter�sticas clonables. \\

La labor a desarrollar dentro del marco concreto de este proyecto de fin carrera fue continuar el proyecto Hydra donde se hab�a dejado anteriormente, es decir, una vez los objetivos 1 y 3 hab�an sido cumplimentados, pasar a implementar la funcionalidad correspondiente a los objetivos 2 y 4. Para satisfacer dichos objetivos, se realizaron las tareas que se describen a continuaci�n: \\

1. Estudio del estado del arte. El objetivo de esta fase es adquirir los conceptos necesarios para comprender el contexto del proyecto Hydra, as� como los necesarios para continuar desarrollando la aplicaci�n en el punto en que fue visitada por �ltima vez. M�s concretamente, ha sido fundamental familiarizarse con los conceptos de L�nea de Producto Software, �rbol de Caracter�sticas (con y sin caracter�sticas clonables) y de Ingenier�a Dirigida por Modelos en general, y de Ingenier�a de Lenguajes Dirigida por Modelos en particular.

2. Estudio de las herramientas utilizadas. El objetivo de esta fase comprende la familiarizaci�n con todas las herramientas y tecnolog�as necesarias para desarrollar la parte estipulada de la aplicaci�n. En concreto, con EMF, Ecore, EMFText, Eclipse Validation Framework, Eclipse Plugin Development y Subversion.

3. Desarrollo de un editor de restricciones externas entre caracter�sticas. El objetivo de este editor es soportar la especificaci�n de restricciones externas ante un modelo de caracter�sticas proporcionado por el usuario. Tales restricciones son expresiones similares a f�rmulas l�gicas, salvo por alguna peculiaridad espec�fica. Es por eso que se opt� por el uso de un editor textual en lugar de uno gr�fico, ya que es el m�todo m�s habitual de representar este tipo de operaciones. Para crear el metamodelo del lenguaje se ha utilizado la herramienta Ecore, mientras que para definir la gram�tica se ha utilizado EMFText. 

4. Desarrollo de un validador de configuraciones. Una vez se finaliz� de crear el editor para las restricciones, el siguiente paso l�gico era aportarle una sem�ntica que permitiera comprobar si las restricciones creadas satisfacen la configuraci�n proporcionada por el usuario. Para implementar la sem�ntica se utilizaron las herramientas EMF, Eclipse Validation Framework y Eclipse Plugin Development. 

5. Validaci�n y pruebas. Con objeto de evaluar, probar y verificar el correcto funcionamiento de nuestra herramienta se han sometido algunas configuraciones del �rbol de caracter�sitcas Smarthome a una serie de pruebas de caja negra, tratando de probar todas las operaciones de restricciones posibles en todos los contextos problem�ticos y habituales.  


\section{Estructura del Documento}
\label{sec:intr:organization}

Tras este cap�tulo de introducci�n, la memoria se estructura tal y como se describe a continuaci�n: El cap�tulo 2 introduce un poco m�s en profundidad los conceptos implicados en la herramienta Hydra, as� como las herramientas m�s determinantes a la hora de llevar a cabo su implementaci�n. El cap�tulo 3 describe la planificaci�n del proyecto desde el punto de vista de las tareas involucradas. El cap�tulo 4 describe en profundidad la parte correspondiente a la creaci�n de las sintaxis concreta y abstracta. El cap�tulo 5 describe el resto de tareas que se han llevado a cabo en el proyecto, y el cap�tulo 6 describe mis conclusiones personales y la situaci�n de la herramienta de cara al futuro.



