%%==================================================================%%
%% Author : Tejedo González, Daniel                                 %%
%%          Sánchez Barreiro, Pablo                                 %%
%% Version: 1.0, 27/11/2012                                         %%
%% Version: 2.0, 09/03/2013                                         %%
%%                                                                  %%
%% Memoria del Proyecto Fin de Carrera                              %%
%% Gramática, Introduccion                                          %%
%%==================================================================%%


De acuerdo con la metodolog��a impuesta por la Ingenier�a de Lenguajes Dirigida por Modelos, el siguiente paso para desarrollar el lenguaje HCL pasa por desarrollar una sintaxis textual que sea capaz de vincular el metamodelo construido con las l�neas de c�digo que escribamos en nuestro lenguaje.  Esta sintaxis textual se construye a trav�s de una gram�tica, en la que especificamos una serie de reglas para cada una de las metaclases, de modo que esas reglas satisfagan la sintaxis y los requisitos de nuestro lenguaje. 

%%==================================================================%%
%% NOTA(Pablo): Cuando llegues a la parte en la cual hablas de      %%
%%              requisitos insertas esto                            %%
%%==================================================================%%

Los requisitos que nuestra gram�tica debe cumplir ya estaban pr�cticamente definidos, pues la gram�tica BNF definida por el profesor Pablo S�nchez, mostrada en la Figura~\ref{fig:constraintBNF}, es m�s o menos la misma que hemos tenido que implementar (con la salvedad que a�ade cambiar de lenguaje BNF a EMFText).

No obstante, hubo que a�adir una serie de consideraciones adicionales para refinar dicha gram�tica. Concretamente, tuvimos que a�adir los siguientes requisitos a la notaci�n BNF de nuestra gram�tica:

\begin{itemize}
    \item Deb�a permitirse la posibilidad de especificar prioridad en las operaciones, es decir, de poder delimitar las operaciones con par�ntesis que denoten el orden de realizaci�n de las mismas.
    \item Todo fichero de especificaci�n de restricciones debe comenzar con una l�nea que indique la ruta donde se encuentra el �rbol de caracter�sticas al que han de aplicarse las restricciones. La sintaxis de este aspecto ser� $import ruta$, donde ruta es una direcci�n que indique un fichero en nuestro computador.
    \item Todas las restricciones han de separarse entre ellas mediante el car�cter `\texttt{;}'.
\end{itemize}

%%==================================================================%%
%% NOTA(Pablo): y luego sigue con la introducción                   %%
%%==================================================================%%

Para hablar de la gram�tica, el cap�tulo se estructurar� del siguiente modo: Tras esta introducci�n, habr� una secci�n en la que se hablar� de la implementaci�n de la gram�tica. Se explicar� paso a paso cada una de las l�neas que la componen, adem�s de su relaci�n con el metamodelo. A continuaci�n, se hablar� en otra secci�n de las pruebas a las que fue sometida la gram�tica para corroborar su correcto funcionamiento.