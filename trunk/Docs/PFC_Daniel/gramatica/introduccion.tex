%%==================================================================%%
%% Author : Tejedo Gonz�lez, Daniel                                 %%
%%          S�nchez Barreiro, Pablo                                 %%
%% Version: 1.0, 27/11/2012                                         %%                   
%% Version: 2.0, 09/02/2013                                         %%                   
%%                                                                  %%
%% Memoria del Proyecto Fin de Carrera                              %%
%% Gram�tica, Introducci�n                                                %%
%%==================================================================%%

Siguiendo la metodolog�a impuesta por la Ingenier�a de Lenguajes Dirigida por Modelos, la siguiente parte del desarrollo de nuestro lenguaje pasa por definir la sintaxis concreta textual del mismo. Para poder llevar a cabo esta tarea es necesario escribir una gram�tica, que se encargar� de transformar el c�digo que escribamos en nuestro lenguaje en instancias v�lidas del metamodelo inicial. La captura de requisitos de la gram�tica es pr�cticamente id�ntica a la que se hizo en el cap�tulo anterior para dise�ar el metamodelo, pues en ambos casos es necesario conocer las operaciones del lenguaje y su sintaxis. 

El cap�tulo se estrucurar� del siguiente modo: en primer lugar hablaremos sin entrar en mucho detalle de la herramienta \emph{EMFText}, que es la que hemos escogido para dise�ar nuestra gram�tica. A continuaci�n, y en la misma secci�n, describiremos esta gram�tica y las partes que la componen. Por �ltimo mostraremos las pruebas a las que fue sometida para corroborar que su funcionamiento era el apropiado.