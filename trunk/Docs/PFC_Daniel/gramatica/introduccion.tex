%%==================================================================%%
%% Author : Tejedo González, Daniel                                 %%
%%          Sánchez Barreiro, Pablo                                 %%
%% Version: 1.0, 27/11/2012                                         %%
%% Version: 2.0, 09/03/2013                                         %%
%%                                                                  %%
%% Memoria del Proyecto Fin de Carrera                              %%
%% Gramática, Introduccion                                          %%
%%==================================================================%%


De acuerdo con la metodología expuesta en \todo{Completar}

%%==================================================================%%
%% NOTA(Pablo): Cuando llegues a la parte en la cual hablas de      %%
%%              requisitos insertas esto                            %%
%%==================================================================%%

Los requisitos que nuestra gramática debía cumplir ya estaban prácticamente definidos, pues la gramática BNF definida por el profesor Pablo Sánchez, mostrada en la Figura~\ref{fig:constraintBNF} prácticamente fijaba la misma.

No obstante, hubo que añadir una serie de consideraciones adicionales para refinar dicha gramática. Concretamente, tuvimos que añadir los siguientes requisitos a la notación BNF de nuestra gramática:

\begin{itemize}
    \item Debía permitirse la posibilidad de especificar prioridad en las operaciones, es decir, de poder delimitar las operaciones con paréntesis que denoten el orden de realización de las mismas.
    \item Todo fichero de especificación de restricciones debe comenzar con una línea que indique la ruta donde se encuentra el árbol de características al que han de aplicarse las restricciones. La sintaxis de este aspecto será $import ruta$, donde ruta es una dirección que indique un fichero en nuestro computador.
    \item Todas las restricciones han de separarse entre ellas mediante el carácter `\texttt{;}'.
\end{itemize}

%%==================================================================%%
%% NOTA(Pablo): y luego sigue con la introducción                   %%
%%==================================================================%%
