%%==================================================================%%
%% Author : Tejedo Gonz�lez, Daniel                                 %%
%%          S�nchez Barreiro, Pablo                                 %%
%% Version: 1.0, 25/11/2012                                         %%
%% Version: 1.0, 06/02/2013                                         %%
%%                                                                  %%
%% Memoria del Proyecto Fin de Carrera                              %%
%% Sintaxis abstracta,  introducci�n                                     %%
%%==================================================================%%

Siguiendo la metodolog�a impuesta por la Ingenier�a de Lenguajes Dirigida por Modelos, la primera parte del desarrollo de nuestro lenguaje pasa por definir un metamodelo que represente la sintaxis abstracta del mismo. Este cap�tulo trata de mostrar con cierto nivel de detalle el proceso de elaboraci�n de esa tarea, desde su concepci�n hasta su implementaci�n final.

Para ello el cap�tulo se estructurar� del siguiente modo: en primer lugar contar� con una secci�n de captura de requisitos, donde se se�alar�n con claridad las funcionalidades y restricciones que debe poseer el lenguaje que queremos crear. Una vez que eso ha quedado claro, se pasar� a describir en detalle la tarea de creaci�n del metamodelo, describiendo por encima cada una de las partes que lo componen y su estructura. A continuaci�n se proceder� a explicar el proceso de especificaci�n de restricciones externas, durante el cual se implementar�n ciertos aspectos del lenguaje imposibles de representar �nicamente con el metamodelo. Para finalizar, se incluir� una secci�n con las pruebas a las que hemos sometido la sintaxis abstracta para corroborar su correcto funcionamiento.