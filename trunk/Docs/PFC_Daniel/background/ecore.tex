%%==================================================================%%
%% Author : Tejedo Gonz�lez, Daniel                                 %%
%%          S�nchez Barreiro, Pablo                                 %%
%% Version: 1.0, 18/11/2012                                         %%                   %%                                                                  %%
%% Memoria del Proyecto Fin de Carrera                              %%
%% Antecedentes, ecore                                       %%
%%==================================================================%%

EMF o Eclipse Modeling Framework es un plug-in para eclipse que permite trabajar con la metodol�a de Ingenier�a Dirigida por Modelos. Tiene incorporado varios generadores de c�digo que permiten, entre otra multitud de funciones, crear autom�ticamente la implementaci�n de los modelos que creemos, as� como su integraci�n inmediata como plug-in con el entorno eclipse. Adem�s, es capaz de integrarse con multitud de herramientas orientadas a tareas mucho m�s espec�ficas dentro de la Ingenier�a Dirigida por Modelos, de las cuales cabe destacar Ecore.

Ecore es la herramienta que permite la creaci�n y edici�n de metamodelos, gracias a un entorno visual bastante atractivo que facilita enormemente el proceso. El fichero resultante de nuestra creaci�n presentar� autom�ticamente un formato est�ndar XMI, que es el utilizado para todo tipo de modelos y sus instancias. Ecore posee otro tipo de funcionalidades menos utilizadas que se engloban dentro del paquete Ecore Tools, enfocadas todas ellas al uso de los metamodelos y su validaci�n.

Por �ltimo, EMF tambi�n incorpora una herramienta integrada con Ecore para la validaci�n de las m�ltiples sintaxis concretas que podamos construir. EMF Validation Framework sirve, en otras palabras, como soporte en caso de que nuestro lenguaje pueda contener reglas adicionales que no puedan ser satisfechas �nicamente con la descripci�n del metamodelo y la gram�tica. En el caso particular de nuestro editor para escificaci�n y validaci�n de restricciones hemos utilizado EMF Validation Framework para comprobar si las caracter�sticas escritas por el usuario existen en el modelo importado, y tambi�n para determinar si tienen cardinalidad o no.