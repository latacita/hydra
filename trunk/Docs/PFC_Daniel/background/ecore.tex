%%==================================================================%%
%% Author : Tejedo Gonz�lez, Daniel                                 %%
%%          S�nchez Barreiro, Pablo                                 %%
%% Version: 1.0, 18/11/2012                                         %%                  
%%                                                                  %%
%% Memoria del Proyecto Fin de Carrera                              %%
%% Antecedentes, ecore                                              %%
%%==================================================================%%

EMF \emph{Eclipse Modeling Framework}~\cite{steinberg:2008} es un \emph{plug-in} para Eclipse~\cite{clayberg:2008} que permite elaborar metamodelos. Pera ello proporciona un lenguaje de metamodelado denominado Ecore, el cual se ha convertido en el est�ndar \emph{de facto} para la realizaci�n de metamodelos. Utilizando Ecore se pueden crear metamodelos de forma gr�fica usando una notaci�n muy similar a la los diagramas de clases de UML. La Figura~\ref{fig:sle:metamodeloGrafo} muestra un sencillo ejemplo de metamodelo en Ecore (ver Secci�n~\ref{sec:intr:sle} para m�s detalles). EMF tambi�n incorpora una herramienta para la validaci�n reglas adicionales que no puedan ser especificadas a nivel de del metamodelo. 
 
EMF permite que, a partir de un metamodelo especificado en Ecore, podamos, utilizando diversos generadores de c�digo, crear autom�ticamente un conjunto de clases que nos permiten manipular dichos modelos a nivel de c�digo. Dichas clases se pueden adem�s distribuir como \emph{plug-in} para el entorno Eclipse.

Adem�s, al haberse convertido en est�ndar \emph{de facto} para el desarrollo de metamodelos, Ecore es compatible con multitud de herramientas para Ingenier�a de Lenguajes Dirigida por Modelos, como EMFText, la cual se describe en la siguiente secci�n, o diversos generadores de c�digo o herramientas de transformaci�n de modelos. 

