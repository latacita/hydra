%%==================================================================%%
%% Author : Tejedo Gonz�lez, Daniel                                 %%
%%          S�nchez Barreiro, Pablo                                 %%
%% Version: 1.0, 18/11/2012                                         %%                   %%                                                                  %%
%% Memoria del Proyecto Fin de Carrera                              %%
%% Antecedentes, arquitectura de plugins de eclipse                                      %%
%%==================================================================%%

Un plug-in en Eclipse es un componente que provee un cierto tipo de servicio dentro del contexto del espacio de trabajo de eclipse, es decir, una herramienta que se puede integrar en el entorno Eclipse junto con sus otras funcionalidades. Dado que la herramienta Hydra fue dise�ada como un plug-in para Eclipse, y nuestro editor es una parte de la misma, ha sido necesario aprender el manejo de algunas de las funcionalidades de la arquitectura de plug-ins de Eclipse.

En particular, se han utilizado mucho los puntos de extensi�n. Un punto de extensi�n en un plug-in indica la posibilidad de que ese plug-in sea a su vez parte de otro, o que haya otros plug-ins que sean parte de �l. Esta particularidad permite no s�lo la integraci�n de nuestro editor con Hydra, sino tambi�n la personalizaci�n de men�s y botones para �l gracias a la creaci�n de puntos de extensi�n con plug-ins de creaci�n de men�s y barras de herramientas.