%=============================================================================%
% Author: Pablo S�nchez                                                       %
%        (p.sanchez@unican.es http://personales.unican.es/sanchezbp)          %                                                                               %                                                                             %
% Section : Main File                                      Date: 23/02/2011   %
% Version : 1.0                                                               %                                                                             %                                                                             %
% Conference: SPLC 2011                                                       %
%=============================================================================%

\documentclass[10pt, conference, compsocconf]{IEEEtran}

\IEEEoverridecommandlockouts

\usepackage{graphicx}
\usepackage{amsfonts}
\usepackage{color}
\usepackage{dsfont}
\usepackage{url}


\newcommand{\imp}[1]{{\small \sf #1}}
\newcommand{\todo}[1]{\color{red} TODO: #1 \color{black}}
\newcommand{\materialUrl}{\url{http://personales.unican.es/sanchezbp/spl/hydra/splc2011}}

\begin{document}
%
% paper title
% can use linebreaks \\ within to get better formatting as desired
\title{Analysis of Constraints including Clonable Features using Hydra
\thanks{This work has been supported by the Spanish Ministry Project TIN2008-01942/TIN, the EC STREP Project AMPLE IST-033710 and the Junta de Andaluc{\'i}a regional project FamWare TIC-5231}}

\author{
\IEEEauthorblockN{Pablo S\'{a}nchez}
\IEEEauthorblockA{Dpto. Matem\'{a}ticas, Estad\'{i}stica y Computaci\'{o}n  \\
Universidad de Cantabria \\
Santander (Cantabria, Spain) \\
p.sanchez@unican.es}
\and
\IEEEauthorblockN{Jos\'{e} Ram\'{o}n Salazar, Lidia Fuentes}
\IEEEauthorblockA{
Dpto. Lenguajes y Ciencias de la Computaci\'{o}n \\
Universidad de M\'{a}laga \\
M\'{a}laga (M\'{a}laga, Spain) \\
\{salazar, lff\}@lcc.uma.es}
}

\maketitle

\begin{abstract}
	%===============================================================================%
% Author: Pablo S�nchez                                                         %
%         p.sanchez@unican.es                                                   %
%         http://personales.unican.es/sanchezbp                                 %
% Section : Abstract                                           Date: 24/02/2011 %
% Version : 1.0                                                                 %
% Conference: SPLC 2011                                                         %
%===============================================================================%

Clonable features, i.e. features whose cardinality has an upper bound greater than 1, were proposed several years ago. Although they have been included in several feature modelling tools, there is still no tool able to properly deal with constraints, such as dependencies or mutual exclusions, which include clonable features. The first challenge these tools find is that the semantics of these constraints becomes undefined in the presence of clonable features. As a consequence of this lack of semantics, the analysis and validation of these constraints is simply not feasible. To overcome these limitations, this paper presents: (1) a language for specifying external constraints involving clonable features with a clearly defined semantics; and (2) how to analyse these constraints by transforming them into a Constraint Satisfaction Problem (CSP). This language and the analyser have been incorporated into our feature modelling tool, called \emph{Hydra}. We validate our ideas by applying them to a SmartHome industrial software product line. 
\end{abstract}

\begin{IEEEkeywords}
Feature Models, Clonable Features, Constraint Analysis
\end{IEEEkeywords}

\section{Introduction}
\label{sec:introduction}

%==================================================================%
% Author : Doe Doe, John                                           %
%          S�nchez Barreiro, Pablo                                 %
% Version: 1.0, dd/mm/yyyy                                         %                   %                                                                  %
% Memoria del Proyecto Fin de Carrera                              %
% Introducci�n, archivo ra�z                                       %
%==================================================================%

%%% Schema to write a paper introduction
%% Description of Purpose
	% What problem, issue or question does this research address ?
		%
	% What limitations or failings of current understanding, knowledge, method,
	% or technologies does this research resolve ?
		%
	% What is the significance of the problem issue or question ?
		%
%% Goal statement
	% What new understanding, knowledge, methods or technologies will this
	% research generate ?
		%
	% How this address the purpose of the work ?
		%
%% Approach
	% What experiments, prototypes or studies will be done to achieve the stated % goal ?
		%
	% How will achievement or contribution of the research be demonstrated or validated ?
		%

\chapterheader{Introduction}{Introduction}
\label{chap:introduction}

% Introducci�n al cap�tulo

\chaptertoc

\section{Introducci�n}
\label{sec:intr:introduction}

TODO: Siguiendo el esquema que aparece arriba, escribir la introducci�n

\section{Motivaci�n and Contribuciones}
\label{sec:intr:motivation}

TODO: Esta secci�n es m�s para tesis doctorales que para proyectos fin de carrera. La dejamos de momento pero se podr�a eliminar

\section{Visi�n General del Proyecto}
\label{sec:intr:overview}

TODO: Esto est� bien dejarlo, pero tambi�n es suprimible

\section{Estructura del Documento}
\label{sec:intr:organization}

Esto es una especie de �ndice ampliado y se deja, suele ser bastante �til para que el que est� vago se lea esto y se acabe el problema.







\section{Motivation}
\label{sec:motivation}

%============================================================================%
% Author: Pablo S�nchez                                                      %
%         p.sanchez@unican.es, http://personales.unican.es/sanchezbp         %
% Section : Motivation                                     Date: 25/02/2011  %
% Version : 1.0                                                              %
% Conference: SPLC 2011                                                      %
%============================================================================%

%============================================================================
% NOTE(Pablo): The angle of this section has changed 
%============================================================================
%
% This section firstly introduces clonable 
% features~\cite{czarnecki:2005d,czarnecki:2005} and explains the concepts 
% and ideas behind them. Then, we explain why clonable features are 
% important. Finally, we identify the research challenges created by 
% clonable features.
%
%============================================================================

This section firstly introduces the current shortcomings of feature modelling tools we faced during the development of a SmartHome SPL in the context of the AMPLE project. Next, we comment on the research challenges we discovered and we aim to solve in this paper.

\subsection{Current shortcomings related to clonable features}

\input{shortcomings}

\subsection{Research challenges on clonable features}

%============================================================================%
% Author: Pablo S�nchez                                                      %
%         p.sanchez@unican.es, http://personales.unican.es/sanchezbp         %
% Section : Research Challenges                            Date: 25/02/2011  %
% Version : 1.0                                                              %
% Conference: SPLC 2011                                                      %
%============================================================================%

%============================================================================================
% NOTE(Pablo): This is not required for SPLC 2011
%============================================================================================
%
% The creation of clonable features was initially an easy task, since it only required to add
% the notion of cardinality to each feature. Nevertheless, this has created several
% side-effects, since some concepts such as the semantics of a clonable feature selection, had % to be reviewed and updated. This section identifies several problems, not currently solved to % the best of our knowledge, regarding the specification of external constraints involving
% clonable features.
%
%============================================================================================

%============================================================================================
% NOTE(Pablo): Too much verbose
%============================================================================================
%
% Nevertheless, this had several consequences, since it was required to review an update
% well-established concepts related to feature modelling and to answer several research
% challenges that emerged as a consequence of introducing this new concept. Previous section
% described how clonable features are selected by means of a new cloning operation. This
% section identifies several problems, not currently solved to the best of our knowledge,
% regarding the specification of external constraints between features when some of the
% features involved in the constraint are clonable.
%
%============================================================================================

Clonable features creates new challenges when the specification of cross-tree constraints is required. For instance, according to Figure~\ref{fig:smartHomeFM}, house facilities can be selected at a house, floor or room level. Therefore, a cross-tree constraint should specify if \imp{LightMng} has been selected at the floor level, this facility must also have been selected for each room belonging to that floor.

Since \imp{SmartEnergyMng} requires the \imp{HeaterMng} and \imp{WindowMng} features have been selected, other constraint should specify whenever \imp{SmartEnergyMng} has been selected at one level (e.g. for a specific floor), \imp{HeaterMng} and \imp{WindowMng} have also been selected at the same level (i.e. for that specific floor). Finally, we might want to specify more advanced constraints, like if \imp{Presence Simulation} is going to be included in a automated house, a 30\% of the rooms in the house at least must include automatic light management .

In feature models, the most usual way to express cross-tree constraints is using propositional logic formulas, such as \imp{SmartEnergyMng implies (LightMng and HeaterMng)}, where the different features are the atoms for these formulas~\cite{batory:2005}. An atom is evaluated to true when the corresponding feature has been selected, otherwise it is false. The main problem when dealing with clonable features is that this correspondence cannot be used. Clonable features are not selected or unselected. Clonable are just cloned. So, the meaning of a atom related to a clonable feature becomes undefined. As a result, a propositional logical formula including clonable features can not be evaluated and it can not be determined when the corresponding constraint has been satisfied or violated.

For instance, let us suppose  \imp{A} and \imp{B} are clonable features. In this case, what would be the meaning of a external constraint like \imp{A implies B}? Does this means that one instance of \imp{A} implies the existence of at least one instance of \imp{B}? Or, on the other hand, does it means that the existence of all potential \imp{A}s implies the existence of all potential \imp{B}s? Or, why not, the existence of all potential \imp{A}s implies the existence of only one \imp{B}? So, the first research challenge we face is to decide what a clonable feature exactly means in a logical formula expressing a cross-tree constraint.

Each clonable feature really evaluates to a set of clones. For instance, the feature \imp{Floor} in Figure~\ref{fig:smartHomeFM} actually evaluates to \imp{\{Ground, First \}}, instead of selected or unselected. Thus, we might want to specify properties that only applies to: (1) at least one element in that set (e.g. only to the \imp{Ground} floor); (2) to all the elements in that set that fulfill a certain constraint (e.g to all the floors with \imp{LightMng} selected in one room at least); or (3) all the elements in the set. This would allow us to express more complex constraints such as, such as: if \imp{HeaterMng} is selected per house level, one \imp{Room} at least must contain one \imp{Heater} as a minimum. So, our second research challenge is to add quantification mechanisms to constraints involving clonable features.

%============================================================================================
% TODO(Pablo): Redo this image using Hydra. It is quite hard to get Hydra working for this
%              purpose        
%============================================================================================

\begin{figure}[!tb]
  % Requires \usepackage{graphicx}
  \centering \includegraphics[width=\linewidth]{Figures/contexts.eps} \\
  \caption{(Left) Invalid configuration (Right) Valid configuration}
  \label{fig:contexts}
\end{figure}

Moreover, as already commented in previous section, we might want to specify constraints which must be evaluated for a particular subtree of the whole feature model, i.e. in a particular \emph{context}. For instance, if \imp{LightMng} has been selected for a particular \imp{Room}, such a \imp{Room} must have one \imp{Light} at least. Thus, we should specify a constraint like ``\imp{LightMng} implies one \imp{Light} device at least''. If we do not limit the scope in which this constraint is evaluated, this constraint will be true for the configurations of Figure~\ref{fig:contexts} (a) and (b). Nevertheless, this constraint should be false for Figure~\ref{fig:contexts} (a), since \imp{r1:Room} has selected the \imp{LightMng} facility, but it has not any \imp{Light} device to control. This means this constraint must be evaluated for all rooms (notice we are using again quantification) and using only the subtree below each \imp{Room}. Thus, the third research challenge is how to specify and analyse constraints which must be must be evaluated in the scope of a particular context.

%============================================================================================
% NOTE(Pablo): Since we have opted for leaving out feature references, this paragraph must
%              also be left out of the paper.
%============================================================================================
%
% Contexts are also useful for solve ambiguities when using feature references, since multiple % copies of a same feature might appear at different part of a configuration model. For 
% instance, does \imp{LightMng} refers to \imp{LightMng} for \imp{GeneralFacilities}, 
% \imp{FloorFacilities} or \imp{RoomFacilities}. Using contexts, we can limit the scope of a 
% name to unambiguously refer to a certain feature.
%
%============================================================================================

Finally, we would to like to point out that not only clonable features can have more than one instance per configuration of a feature model. All those features that have as an ancestor a clonable features can appear more than once in a configuration model. We will call to this kind of feature, i.e. non clonable features with a clonable ancestor, \emph{multiple features}. 
Figure~\ref{fig:multipleFtr} illustrates this situation.In this case, the feature \imp{LightMng} below \imp{RoomFacilities} appears two times due to the \imp{Room} feature has been cloned twice. This implies that \imp{LightMng}, as well as the other features below \imp{Room}, also evaluate to a set of features instead of simply to true or false. Nevertheless, it should be noticed that these multiple features, oppositely to clonable features, can be evaluated to true or false if they are evaluated in a particular context.
For example, the \imp{LightMng} feature in Figure~\ref{fig:multipleFtr} can be evaluated to true or false when it is evaluated in the context of a particular room. So, our last research challenge is how to deal appropriately with \emph{multiple features}.

\begin{figure}[!tb]
  \centering \includegraphics[width=.65\linewidth]{Figures/multipleFeatures.eps} \\
  \caption{The \emph{multiple features} phenomena}
  \label{fig:multipleFtr}
\end{figure}

Summarising, when dealing with clonable features, we should to address the following research challenges to properly express and analyse cross-tree constraints between features:

\begin{enumerate}
	\item What a clonable feature means in a cross-tree constraint.
	\item Add quantification mechanisms to cross-tree constraints.
	\item Add the notion of \emph{context} to cross-tree constraints.
	\item Manage properly multiple features.
\end{enumerate}

To the best of our knowledge, there is currently no research work, and consequently no tool, addressing these challenges. To overcome this limitation, we have firstly created a language for expressing arbitrary complex constraints including clonable features as a stepping stone towards the analysis of constraints including clonable (or multiple) features. Next section describes sauch a language.

%============================================================================================
% NOTE(Pablo) : This have been supressed because it sounds redundant
%============================================================================================
%
% We have implemented a reasoner able to decide if a set of external constraints, involving
% clonable or multiple features, is satisfied given a specific configuration of a feature 
% model. The reasoner is also able to perform some extra task, such as deciding which features % must be incorporated to a configuration in order to satisfy the external constraints. We have % integrated this language and the reasoner in our feature modelling environment, which we have % called \emph{Hydra}.
%
%============================================================================================

%============================================================================================
% NOTE(Pablo) : We have reduced this paragraph. We are no presenting a tool, so this 
%               paragraph can be simply skipped
%============================================================================================
%
% State-of-art tools only offer two operators, \imp{implies} and \imp{excludes}, and deal with % simple propositional formulas. This is clearly not enough to address the research challenges % described in this section.
%
%============================================================================================

%=============================================================================================
% NOTE(Pablo): This paragraph would be suitable for a very general venue                                                                                 %=============================================================================================
%
% A typical feature modelling process would be as follows:
%
% \begin{enumerate}
%    \item First of all, a feature model is created. A feature model is a tree representation 
%          of the features included in % a set of products and of the relationships between 
%          them.
%    \item Since not all the relationships between features can be captured using simply the 
%          syntax provided by feature  models, it is often required to specify some external 
%          constraints, which restrict the way in which features in a feature model can be 
%          selected. For instance, we might specify that if a certain feature \imp{A} is 
%          selected, another feature \imp{B} can not be selected, due to a bad interaction 
%          between these features.
%    \item Once we have created a feature model, we can use it for creating configurations, 
%          i.e. selection of features that specify which particular features are, or it will 
%          be, included in a certain product.
%    \item To ensure we are creating correct configurations, we need to validate that a 
%          configuration obeys the rules the % syntax of the feature model, and, in addition, 
%          it also satisfies the external defined constraints.
% \end{enumerate}
%
%=============================================================================================



%========================================================================%
% NOTA : Perhaps it is better to move this section at the end, mainly    %
%        because to explain why our tool improves the state-of-art, we   %
%        need to                                                         %
%========================================================================%

\section{A language for expressing constraints including clonable/multiple features}
\label{sec:theLanguage}

%============================================================================%
% Author: Pablo S�nchez                                                      %
%         p.sanchez@unican.es, http://personales.unican.es/sanchezbp         %
% Section : The language                                   Date: 28/02/2011  %
% Version : 1.0                                                              %
% Conference: SPLC 2011                                                      %
%============================================================================%

% \subsection{Expressing external constraints with clonable features}
% \input{expressing}

This section presents the language we propose for expressing constraints on feature models including clonable features. Figure~\ref{fig:languageSyntax} shows the syntax, in EBNF notation, for such a language.

\begin{figure*}
	\begin{center}
	\begin{footnotesize}
	\begin{verbatim}
00 <Constraint> ::= "true" | "false" | <SimpleFeature> | <Constraint> <BinaryOp> <Constraint> |
01                  <UnaryOp> <Constraint> | "(" <Constraint> ")" | <ContextExp> | <ComparisonExp>;
02 <BinaryOp>   ::= "and" | "or" | "xor" | "implies";
03 <UnaryOp>    ::= "not";
04 <ContextExp> ::= <SimpleFtr> "[" <Constraint> "]" | "all" <MultiValueFtr> "[" <Constraint> "]" |
05                  "any" <MultiValueFtr> "[" <Constraint> "]";
06 <ComparisonExp> ::= <NumericalExp> <ComparisonOp> <NumericalExp>;
07 <ComparisonOp>  ::= "<" | "<=" | "=" | "=>" | ">" | "!=";
08 <NumericalExp>  ::= <MultiValueFtr> | "SimpleArithmeticExp" | <MultiValueFtr> "[" <Constraint> "]";
	\end{verbatim}
	\end{footnotesize}
	\end{center}
	\caption{Syntax of the Hydra Constraint Language}
	\label{fig:languageSyntax}
\end{figure*}

A \emph{constraint} is a logical expression that evaluates to true or false. A constraint can be simply a literal, i.e \imp{true} or \imp{false} (Figure~\ref{fig:languageSyntax} line 00), which evaluates to true and false, respectively. A constraint can also be a simple feature, i.e. a feature that can appear only once as a maximum in a given context. A \imp{SimpleFeature} evaluates to true if it is selected, otherwise, it evaluates to false.

Clonable features and multiple features are represented in the syntax as \imp{MultiValueFtr}.
A \imp{MultiValueFtr} evaluates to a positive integer (zero included). This positive integer represents the number of clones of that feature contained in a given context of a  configuration model. Since they are numbers, we can use comparison operators, more specifically $<, <=, =, >=, >$ to construct comparison expressions on the number of existing clones. In addition, we can also use these numerical values on basic arithmetic expression, i.e. we can sum, substract, multiply and divide these numerical values. These arithmetic expressions can be used as subexpressions, or operands, in comparison expressions. The comparison expressions evaluate to true or false. Thus, we can use a comparison expression as a subexpression of a more complex logical expression. This solves the first challenge we identified in previous section, which was how to evaluate clonable and multiple features.

Using the language of Figure~\ref{fig:languageSyntax}, the context to evaluate a constraint can also be specified. This can be made in several ways. A context can be specified by surrounding a constraint with brackets and given the name of a feature at the beginning of that expression (Figure~\ref{fig:languageSyntax}, lines 04-05, line 08 at the end). The feature used as context can be a simple feature or, instead, it can be a clonable/multiple feature. In the first case, the constraint placed into brackets is evaluated using the subtree of the configuration model with root in the simple feature used as context.

In this second case, we can use the operators \imp{all} and \imp{any}. If \imp{all} is used,
the constraint enclosed in brackets must be true for all the instances of the \imp{MultiValueFtr} used as context for the expression being true. Otherwise, it evaluates to false. If \imp{any} is used, the constraint enclosed in brackets must be true for one instance of the \imp{MultiValueFtr} used as context at least. If such an instance does not exist, the constraint expression evaluates to false. For instance, \imp{any Room[LightMng]} would be evaluated to true for the configuration model of Figure~\ref{fig:smartHomeCfg}, whereas \imp{all Room[LightMng]} would be evaluated to false. This solves the second research challenge identified in the previous section, which was how to deal with quantification mechanisms.

Moreover, none of these operators might be used. In this latter case, the context expression evaluates to the number of instances of the \imp{MultiValueFtr} feature used as context for which the enclosed constraint is true.  For instance, the context expression \imp{Room[LightMng]} would evaluate to 2 for the configuration model depicted in Figure~\ref{fig:smartHomeCfg}, since \imp{LightMng} has been selected in two rooms, more specifically, in the \imp{Kitchen} and in the \imp{Bedroom}. This, as well as the contents of the previous paragraph, solve the third challenge described in the previous section, which was how to deal with contexts.

%============================================================================================
% NOTE(Pablo) : This is not of interest for the purpose of the paper
%============================================================================================
%
% Our feature modelling tool, called \emph{Hydra} checks that each name refers exclusively to
% only one feature. If different features share the same name in the feature model, they need
% to be disambiguated using contexts.
%
%============================================================================================

We would like to highlight that a feature can be simple in a given context and multiple in another context. For instance, \imp{LightMng} (see Figure~\ref{fig:smartHomeFM}) is simple in the context of \imp{GeneralFacilities} and multiple in the context of \imp{SmartHome}. This means that \imp{LightMng}, according to our syntax, can be used as a \imp{SimpleFtr} inside the \imp{Room} context; but it must be used as a \imp{MultiValueFtr} in the \imp{SmartHome} context. Therefore, \imp{Floor[LightMng]} would be not a well-formed constraint, whereas \imp{Room[LightMng]} would be. Therefore, a tool implementing this language should be aware of these details. This would solve the four research challenge identified in the previous section, which was how to deal with multiple features properly. We have taken into account this detail when implementing this language into our feature modelling tool called Hydra.

%============================================================================================
% NOTE(Pablo) : This is not of interest for the purpose of the paper
%============================================================================================
%
% Hydra takes care of this by means  checking of what kind each feature is in a given context. % Basically, a feature is a multivalue feature if: (1) it is a clonable feature; or (2) in a
% given context, one of their ancestors is a clonable feature. Then, Hydra checks we are not
% using multivalue features as terminal symbols and that each multivalue feature is embedded in % a \imp{ComparisonExpression} which returns a boolean value at the end.
%
%============================================================================================

%============================================================================================
% NOTE(Pablo) : This has become redundant
%============================================================================================
%
% Finally, in context expression with multivalue features, we can also use quantification
% operators \imp{all} and \imp{any} to specify the number of clones of that feature for which
% the specified constraint must be true. Context expression using quantifiers evaluate to true % or false. An expression quantified by \imp{any} evaluates to true, if the constraint
% evaluates to true for at least the context provided by one clone of the multivalue feature.
% Otherwise, it evaluates to false. An expression quantified by \imp{all} evaluates to true,
% if the constraint evaluates to true in each contexts provided by all the clones of the
% multivalue feature. Otherwise, it evaluates to false. For instance, \imp{any Room[LightMng]} % would evaluate to true using the configuration model of Figure~\ref{fig:smartHomeCfg},
% whereas \imp{all Room[LightMng]} would evaluate to false. This solves the second research
% challenge identified in the previous section, which was how to deal with quantification
% mechanisms.
%
%============================================================================================

We have validated this language by applying it to the SmartHome case study. Table~\ref{fig:constraints} shows the cross-tree constraints we have added to the feature model of Figure~\ref{fig:smartHomeFM} to avoid creating invalid configurations. We have also applied it to the case studies mentioned in the introduction\footnote{These case studies can be found in \imp{TODO: provide URL}}. 

\imp{C01} specifies that if \imp{SmartEnergyMng} feature is selected for the whole house, the \imp{LightMng} feature and the \imp{WindowMng} must also be selected, since the first one depends on the latter ones. \imp{C05} specifies this constraint must be also satisfied at the floor facilities level for all the floors, and \imp{C10} does the same at the room level. \imp{C01-C04} specify that when a facility is selected for the whole house, it must also be selected for all the floors. \imp{C01-C04} specify the same relationship between the floor and the room levels.
\imp{C11-C13} indicate when a facility is selected, one instances of the corresponding device to be controlled muts be added to the house. Finally, \imp{C14} specifies that for the \imp{PresenceSimulation} working properly, at least a quarter of the house rooms must have automatic control of the lights. It should be noticed that, to the best of our knowledge, this kind of constraint is not supported by any feature modelling tool or language.

% Review, add the corresponding ones to Presence Simulation

\begin{figure*}[!tb]
	\begin{center}
	\begin{footnotesize}
	\begin{verbatim}
C00 GeneralFacilities[SmartEnergyMng implies (HeaterMng and WindowMng)])
C01 GeneralFacilities[LightMng] implies (all FloorFacilities[LightMng])
C02 GeneralFacilities[WindowMng] implies (all FloorFacilities[WindowMng])
C03 GeneralFacilities[HeaterMng] implies (all FloorFacilities[HeaterMng])
C04 GeneralFacilities[SmartEnergy] implies (all Floor[FloorFacilities[SmartEnergy]])
C05 all FloorFacilities[SmartEnergy implies (HeaterMng and WindowMng)])
C06 all Floor[FloorFacilities[LightMng] implies (all Room[LightMng])]
C07 all Floor[FloorFacilities[WindowMng] implies (all Room[WindowMng])]
C08 all Floor[FloorFacilities[HeaterMng] implies (all Room[HeaterMng])]
C09 all Floor[FloorFacilities[SmartEnergy] implies (all Room[SmartEnergy])]
C10 all Room[SmartEnergy implies (HeaterMng and WindowMng)]
C11 all Room[LightMng implies (Light > 0)]
C12 all Room[WindowMng implies (Window > 0)]
C13 all Room[HeaterMng implies (Heater > 0)]
C14 PresenceSimulation implies ((Room[Light] / Room) * 100 >= 25) 
	\end{verbatim}
	\end{footnotesize}
	\end{center}
	\caption{Cross-tree constraints for the Smart Home feature model}
	\label{fig:constraints}
\end{figure*}

Next section describes how we can translate these expression into a Constraint Satisfaction Problem that can be solved using third-party libraries.

%============================================================================
% NOTE(Pablo): I m not happy with this text here. It must be moved
%============================================================================
%
% For instance, we can write a constraint such as illustrated below.
%
% \begin{equation}
% Room >= 3
% \end{equation}
%
% This constraint would specify a business rule that states that automated houses, in order to % be cost-effective, must have at least three rooms. We add to the language the logical
% operators $and, or, not$ and $implies$, with the usual semantics. Using this simple language, % we can express constraints on clonable features such as (1).
%
%============================================================================






\section{Analysis of constraints including clonable/multiple features}
\label{sec:analysis}

%===============================================================================%
% Author: Pablo S�nchez (pablo@lcc.uma.es; http://www.lcc.uma.es/~pablo)        %
% Section : Expressing and validating ...                    Date: 25/11/2009   %
% Version : 1.0                                                                 %
% Conference: Caise 2010                                                        %
%===============================================================================%

Once we have specified a set of external constraints, we need to design a mechanism for evaluating them given a certain input configuration model, in order to decide if such a configuration model satisfies these constraints and it can, therefore, be considered a valid configuration. Moreover, if a configuration were not valid, we would like to know why it is not valid, and, if it is possible, to (semi)automatically carry out some corrective actions. The input configuration model can be a partial configuration model, i.e. a configuration model where certain features has still neither selected nor unselected. 

We can evaluate these constraints and achieve these goals by translating them into a Constraint Satisfaction Problem~\cite{tsang:1993}. A Constraint Satisfaction Problem is defined as a triple $(X, D, C)$, where $X$ is a finite set of variables, $D$ is a finite set of domains of values (one domain for each existing variable in $X$), and $C$ is set of constraints defined on $V$. Thus, we need to decide how many variables we need to create, which domain these variables will have and if we need to adapt our constraints in order to fit in with a constraint satisfaction problem.

A first tentative, is to create a variable by each potential feature, and to assign to each variable a boolean value depending on if the variable has been selected or not. Nevertheless, since clonable features can have an infinite upper bound, there might be an infinite number of variables, and the set of variables $X$ must be finite. But, it should be noticed that although a feature model can have an  infinite number of configurations, each configuration is finite, since each configuration must have, by definition, a finite number of clones. Therefore, clonable and multiple features are translated into variables of a CSP based on a (partial) configuration model, instead of a configuration model. 

The algorithm for creating the variables for the CSP is as follows. We assume that each clone has a unique name, which serves as feature identifier.

\begin{enumerate}
    \item Then, we create a variable for each simple feature in the feature model. The domain of each variable is ${true, false}$. If a feature can be univocally identified, the name of the variable is the same name as the feature. Otherwise, we preclude the name of the variable with as many name of ancestor features were required to univocally identified it. So, a feature referenced as $Facilities[LightMng]$ would be translated into a variable with name \imp{Facilities\_LightMng}
    \item For each clonable or multiple feature, we create a variable with domain ${a..b}$, where $a, b$ are positive integers, and $b$ can be infinite. These boundaries are calculated using the process that will explain below. 
    \item For each clone in the configuration model, we create a variable for each multiple feature that becomes a simple feature in the context of such a clone. The domain of each variable is ${true, false}$. The name of the variable is the same name of the feature, preclude by the name of the clone. So, the variable for the \imp{FloorFacilities} feature belonging to the \imp{GroundFloor} clone would be name as \imp{GroundFloor\_FloorFacilities}. As before, we preclude the name of the variable with as many name of ancestor features were required to univocally identified it.
    \item For each clone, we create a variable for each multiple feature that remains a multiple feature in the context of such a clone. The domain of each variable is ${a..b}$, where $a, b$ are positive integers, and $b$ can be infinite. These boundaries are calculated using the process that will explain below, but using the clone as a root for the feature model.
\end{enumerate}

%==================================================================================================================%
% NOTE(Pablo): I need to update the figure                                                                         %
%==================================================================================================================%

\begin{figure}
  % Requires \usepackage{graphicx}
  \centering \includegraphics[width=.3\linewidth]{Figures/lowerUpperBounds.eps}\\
  \caption{Calculating lower and upper bound of clonable features}
  \label{fig:lowerUpperBounds}
\end{figure}

If a clonable feature has as cardinality ${a..b}$, it does not mean that this feature can have between $a$ and $b$ instances. See for instance Figure~\ref{fig:lowerUpperBounds} (a). Feature \imp{C} has as lower bound $1$, but it means it need to appear at least one time by each feature $B$, and there must be four clones of the feature $B$ at least. Therefore, the minimum number of clones of the feature $C$ in a whole configuration model is four. So, the global lower bound of a feature $F$ is calculated by multiplying the lower bounds of each feature in the path from such a feature $F$ to the root of the feature model. In this path, optional features are considered to have cardinality $0..1$, mandatory features $1..1$ and grouped features has the same cardinality as the feature group. So, the lower bound of feature $C$, in Figure~\ref{fig:lowerUpperBounds} (b) would be zero. Upper bounds are calculated in the same way, but multiplying the upper bounds of each features. For instance, the upper bound of $C$ in Figure~\ref{fig:lowerUpperBounds} (a) would be infinite; and for Figure~\ref{fig:lowerUpperBounds} (a) would be two.

Applying this process to the feature model of Figure~\ref{fig:smartHomeFM} and the configuration model of Figure~\ref{fig:smartHomeCfg} would be as follows:

%=============================================================================================================%
% NOTE(Pablo): Esto es un co�azo y deber�a ser reducible                                                      %
%=============================================================================================================%

\begin{enumerate}
    \item We create a variable for $SmartHome$ and $Facilities$ with domain ${true,false}$.
    \item We create the variables $Floor$, $Room$, $Devices$, $FloorFacilities$, $RoomFacilities$ with domain $[1..*]$.
    \item We create the variables $Window$, $Heater$ and $Light$, $LightMng$,
          $WindowMng$, $HeaterMng$ and $SmartEnergyMng$ with domain $[0..*]$.
    \item We create the variables $Facilities_LightMng$, $Facilities_WindowMng$, $Facilities_LightMng$ and $Facilities_SmartEnergyMng$ with domain ${true,false}$.
    \item We create the variables $GroundFloor_FloorFacilities$, $GroundFloor_FloorFacilities_WindowMng$, $GroundFloor_FloorFacilities_LightMng$ and $GroundFloor_FloorFacilities_SmartEnergyMng$ with domain ${true,false}$.
    \item We create the variables $GroundFloor_Room$, $GroundFloor_RoomFacilities$, $GroundFloor_Devices$ with domain $[1..*]$.
    \item We create the variables $GroundFloor_Window$, $GroundFloor_Heater$, $GroundFloor_Light$, $GroundFloor_LightMng$, $GroundFloor_WindowMng$, $GroundFloor_HeaterMng$ and $GroundFloor_SmartEnergyMng$ with domain $[0..*]$.
    \item We create the variables $Kitchen_RoomFacilities$, $Kitchen_RoomFacilities_WindowMng$, $Kitchen_RoomFacilities_LightMng$ and $GroundFloor_RoomFacilities_SmartEnergyMng$ with domain ${true,false}$.
    \item We create the variables $Kitchen_Window$, $Kitchen_Heater$, $Kitchen_Light$ with domain $[0..*]$.
    \item We create similar variables to steps 8 and 9, but for the \imp{Bedroom} clone.
\end{enumerate}

Then, we translate the different constraints into constraints for a CSP. For solving a CSP, we have opted for using Choco~\cite{}, a Java library for CSP, since it shows a promising performance in several CSP bechmarks~\cite{}. Choco provides logical operators plus comparison operators for specifying constraints. Thus, the problem of translating logical expressions and comparison expressions is reduced to rewriting the constraints specified in the language presented in the previous section in proper Choco syntax. For instance, the constraint \imp{C01} of Figure~\ref{fig:constraints} would be translated into a Choco constraint with the syntax, 

\begin{figure}
\begin{scriptsize}
\begin{verbatim}
    \imp{implies(Facilities\_SmartEnergy,and(Facilities\_HeaterMng,Facilities\_WindowMng))}
\end{verbatim}
\end{scriptsize}
\caption{CSP constraint for constraint \imp{C01}}
\label{eq:c01}
\end{figure}

Thus, the problem is basically reduced to the translation of context expressions with quantifiers. In this case, the solution for translating a constraint like \imp{<quantifier> <MultiValueFeature> [ Constraint ]}, is to replicate the translation of \imp{<Constraint>} as many times as clones the \imp{<MultiValueFeature>} has. The translation of \imp{<Constraint>} is performed as any other constraint, but assuming that each replica of \imp{<Constraint>} is evaluated in the context of an unique clone of such a feature, i.e. the constraint is evaluated using exclusively the subtree below that clone. Therefore, we need to replace the variables in \imp{<Constraint>} by the variables that refer to the features below the clone. If the quantifier is \imp{any}, we join all these replicas by \emph{or} relationships. If the quantifier is \imp{all}, we join all these replicas by \emph{and} relationships.  

% We denote that a constraint $C$ is evaluated using a given feature instance $FI$ as context as \imp{I \vDash C}. Thus, % for instance, using the configuration model of Figure~\ref{fig:smartHomeCfg}, $\imp{Kitchen} $\vDash$  LightMng$} 
% would be evaluated to false, whereas $\imp{Bedroom} $\vDash$ LightMng$} would be evaluated to true.

Thus, for instance, the constraint \imp{C05} in Figure~\ref{fig:constraints}, would be translated following the next process: (1) We calculate the set of clones of the feature $Floor$. In this case, $Floor = {GroundFloor}$. Thus, each feature in the internal constraint refers to a feature below $GroundFloor$; (2) Then, we translate the internal constraint, which generates a Choco constraint such as depicted in~\ref{eq:C05}.

\begin{figure}
\begin{scriptsize}
\begin{verbatim}
  implies(GroundFloor\_FloorFacilities\_SmartEnergy,      
          and(GroundFloor\_FloorFacilities\_HeaterMng,    
              GroundFloor\_FloorFacilities\_WindowMng))}  
\end{verbatim}
\end{scriptsize}
\caption{CSP constraint for constraint \imp{C05}}
\label{eq:c05}
\end{figure}

In the case of the constraint \imp{C13} in Figure~\ref{fig:constraints}, we would need to create several replicas of the constraint, and to use different sets of variables for each replica, according to the different clones of the feature \imp{Room}. The translation of such a constraint is depicted in Figure~\ref{eq:13}

\begin{figure}
\begin{scriptsize}
\begin{verbatim}
  and(implies(Kitchen\_HeaterMng,gt(Kitchen\_Heater,0)),
             implies(Bedroom\_HeaterMng,gt(Bedroom\_Heater,0))))}
\end{verbatim}
\end{scriptsize}
\caption{CSP constraint for constraint \imp{C05}}
\label{eq:c13}
\end{figure}

Finally, we need to bind the variables that have been already selected or unselected in the configuration model. For each boolean variable $v$, a constraint \imp{eq(v,true)} is added to the set of constraints if the corresponding feature has been selected; otherwise, a constraint \imp{eq(v,false)} is added to such a set. For each integer variable representing a clonable or multiple feature, the number of clones is calculated and that variable initialized to that number of clones. 

Once we have defined our CSP, we can solve using a third-party library as Choco. Choco calculates if the current configuration satisfies the external constraints, and if it is not so, it provides information about what constraints has been violated. Moreover, Choco can be used to perform some kind of useful analysis on partial configurations. 

For instance, given an invalid partial configuration, Choco can use constraint propagation to calculate what features should be added to the current configuration in order to create a valid configuration. We can also Choco to complete a configuration given a certain criteria. 

For instance, let us suppose we have the configuration model of Figure~\ref{fig:smartHomeCfg}, but \imp{WindowMng} has not been selected neither at the \imp{Floor} nor at the \imp{Room} level (and it has been selected at the house level). But, according to constraints \imp{C02} and \imp{C07}, it should have been also selected at the floor and room levels. Using constraint propagation, Choco can calculate that these features are lacking at these levels, and \emph{Hydra}, our feature modelling tool, would add them to the configuration model.

If, given a partial configuration, this can be completed in several ways, we can use Choco to select the configuration, which fulfill some kind of arbitrary criteria, such as having the lower number of features. 




% In our case, $V$ would be the features contained in the feature model. Each simple feature is transformed into a
% variable $v \in V$, with domain $D_{i} = {TRUE, FALSE}$. Each clonable and multiple feature is transformed into a
% variable $v \in V$ with domain $D_{i} = {a..b}$, where $a$ is the minimum number of times the clonable feature can
% appear in a feature diagram and $b$ the upper bound. These limits do not necessarily need to be the limits of the
% clonable feature. See for instance, Figure~\ref{fig:lowerUpperBounds} (a). Feature \imp{D} has as lower bound $1$, but % it means it need to appear $1$ time by each feature $B$, and there must be $4$ clones of the feature $B$ at least.
% Therefore, the minimum number of clones of the feature $D$ in a whole configuration model is $4$. The global lower
% bound of a feature $F$ is calculated by multiplying the lower bounds of each feature in the path from such a feature
% $F$ to the root of the feature model. In this path, optional features are considered to have cardinality $0..1$,
% mandatory features $1..1$ and grouped features has the same cardinality as the feature group. Upper bounds are
% calculated in the same way, but multiplying the upper bounds of each features. It should be noticed that any $0$ in
% the lower bound of the features in that path means the lower bound will be zero, and any $*$ in the upper bounds means % that the upper bound will be $*$. Then, for each clone in the feature model we create a variable $v \in V$. We do not % consider clones the selections of simple features. These variables has as domain ${TRUE}$, it is said, they always
% evaluate to true since
% they already exist in the configuration model.



\section{Evaluation and Discussion}
\label{sec:discussion}

%============================================================================%
% Author: Pablo S�nchez                                                      %
%         p.sanchez@unican.es, http://personales.unican.es/sanchezbp         %
% Section : Discussion                                     Date: 01/03/2011  %
% Version : 1.0                                                              %
% Conference: SPLC 2011                                                      %
%============================================================================%

This paper has presented a new language with a well-defined semantics for expressing cross-constraints including clonable features. This language address the four research challenges identified in Section~\ref{sec:motivation}. Our language provides a meaning for clonable/multiple features, quantification operators and contexts. This language has been included  in our feature modelling tool called Hydra. This tool checks that multiple features are correctly used as simple features or multi valued features depending on the context where they appear. 

To validate this language, mainly its expressiveness, we have applied it to three case studies\footnote{The results can be found in \materialUrl}: (1) a SmartHome software product line based on an industrial case study~\cite{ample:d52}; (2) a graphical user interface, based on a domain specific language, presented in Santos et al~\cite{santos:2008}; a (3) the feature model for the satellite communication systems which initially motivated the addition of clonable features to feature models. We have modelled these case studies in Hydra and we have successfully specified the required cross-tree constraints. We would like to point out that the feature model for the Smart Home SPL is a good benchmark for analysing expressiveness of feature modelling tools since it was designed with this purpose at the beginning of the AMPLE project.

Moreover, we have presented in this paper how the constraints expressed in this language, which we have called \emph{Hydra Constraint Language},  can be transformed into a CSP problem to analyse them. Using this technique, we have implemented four analysis operations in Hydra. More specifically, we have implemented the \emph{valid configuration}, \emph{valid partial configuration}, \emph{void feature model}, \emph{dead features} and \emph{promising partial configuration} analysis operations~\cite{Benavides:2010}. We were particularly interested in the \emph{valid partial configuration}, since we were interested in discovering erroneous configurations as early as possible.

As already commented, the reader could argue a Domain-Specific Language (DSL) or Ecore-based model would be more useful for case studies like the Smart Home SPL than feature models with clonable features. Indeed, we explored both paths in the AMPLE project~\cite{ample:product,ample:solution}.

The rationale behind this argumentation is that DSL provides more user-friendly languages for configuring products. For instance, in the SmartHome case study, we might create a DSL (or metamodel) which specified that a automated house can have several floors and inside each floor, several rooms. Then, a visual (or textual) concrete syntax for this DSL might be created and the corresponding tooling developed or, in some cases, semi automatically generated. This visual syntax will clearly depict the containment relationships between floors and rooms, which would increase usability.

Whereas this argumentation is absolutely right, there also some counter argumentations. Fist of all, the visual editor for the DSL has to be developed. Although model-driven tools such as Eclipse GMF (Graphical Modelling Framework) has considerably simplified this task, this development time can not be neglected. Moreover, it requires a good expertise on model-driven techniques. It should also be noticed that a visual editor must be constructed per each case study. If feature models are used, this development time is saved, as designers only need to model a feature model using their favourite modelling tool.

Moreover, where some kind of variability can be nicely expressed using DSLs, other kinds, such as that a Room can optionally have several facilities, are not so nicely expressed. This purely configurative variability is added to DSLs y placing a set of boolean attributes on the DSLs model elements. These attributes are true is the feature has been selected; otherwise they are false. For instance, in the Smart Home case study, our DSL would have \imp{Room} as a model element that can be placed inside a \imp{Floor}. Each instance of \imp{Room} would have as boolean properties \imp{LightMng}, \imp{HeaterMng}, \imp{WindowMng} and \imp{SmartEnergyMng}. The user would have to set these properties to true or false depending on if he or she desires to select them.
Our experience has revealed that most users prefer to use feature model for carrying out this kind of configuration.

Finally, we would like to point out that, as it happens with feature model, no all relationships between elements of a certain domain can be specified in the metamodel that defines a DSL. Therefore, the specification of some external constraints is also required. These constraints are usually specifies using OCL. The main problem with OCL is this is not an easy to learn language. So, most users find difficulties for expressing constraints such as depicted in Figure~\ref{fig:constraints} in OCL. In addition, to carry out analysis operations using a set of constraints expressed in OCL is clearly more difficult than when expressed in other languages. The main reason is that OCL was not designed for supporting this kind of analysis operations, so they are difficult to implement using third-party tools for evaluating OCL expressions. Moreover, OCL has a large metamodel. Thus, to transform a set of OCL constraints into a CSP problem becomes more difficult than use a smaller metamodel, such as the one associated to the Hydra Constraint Language.

In general, we have noticed junior or young software designers used to model-driven technologies often prefer to use DSLs instead of feature models. On the other hand, feature models has been out there for 20 years, so senior software designers with a large experience on feature models are more likely to use clonable features rather than being into a new paradigm and learning new languages and tools, such as Ecore and GMF.

So, summarising, DSLs and feature models with clonable features have both advantages and disadvantages. According to our experience, when DSLs or feature models should be used is, in most of cases, just a matter of taste. Indeed, Ecore models can be transformed into feature models with clonable features, and feature models with clonable features can be transformed into Ecore models~\cite{stephan:2008}. So a possible solution for taking advantage of both solutions would be to use Ecore models as a basis for constructing user-friendly configuration tools and then map these models into feature models which can be analysed using the techniques presented through this paper. Thus, although based on feature models, the work presented in this paper is not incompatible with the use of DSLs.


\section{Related Work}
\label{sec:related}

%===============================================================================%
% Section : Related Work                                     Date: 19/11/2009   %
% Version : 1.0                                                                 %
% Conference: Caise 2010                                                        %
%===============================================================================%

There are currently several tools that supports feature modelling. Table~\ref{} contains a high-level comparison of the most well-known feature modelling tools regarding support for clonable features and usability of the user interface. Most of them supports automatic validation of external constraints defined between features. The common technique used for validating a configuration of a feature model is to transform the feature model and the externally defined constraints


But, as it will be described in this section, there is no tool, to the best of our knowledge, that supports validation of constraints involving clonable features.

%\begin{table}
%    \begin{center}
%    \begin{tabular}{|l|c|c|c|c|}
%        \hline
%        Tool          & Usability &  \multicolumn{3}{c}{Clonable Features}    \\ \hline
%        \multicolumn{2}{|c|}{\ }  &   Modelling  & Configuration & Validation \\ \hline
%        FMP           & Low       &   \checkmark & $\times$      & $\times$   \\ \hline
%        FaMa          &           &   $\times$   & $\times$      & $\times$   \\ \hline
%        Moskkit       & High      &   \checkmark & $\times$      & $\times$   \\ \hline
%        \hline
%    \end{tabular}
%    \end{center}
%\end{table}

%=============================================================================================================%
% NOTE(Pablo): I need to explain some point before that feature models can be expressed as propositional      %
%              formulas                                                                                       %
%=============================================================================================================%

% 2005
Feature Modelling Plugin (FMP)~\cite{czarnecki:2005c} is an Ecore-based~\cite{} Eclipse plugin that supports modelling of cardinality-based feature models. Nevertheless, the configuration of clonable features is not supported at all (tool authors acknowledge that configuration facilities are currently unpredictable). FMP supports the specification of external constraints in the form of propositional logic formulas, but the semantics of clonable features in these constraints are simply unspecified. FMP uses JavaBDD~\cite{} as a back-end for analyzing and validating constraints in feature models. JavaBDD is an open-source Java library for manipulating Binary Decision Diagrams (BDDs), which are used to analyse the satisfiability of a set of propositional formulas, i.e., of a set of feature relationships and external constraints represented as a set of external formulas. Moreover, the Graphical User Interface (GUI) of FMP is based on the default tree-based representation of Ecore Models, which hinders usability and understability of feature models, overall when users need to deal with large scale models.

Feature Model Analyzer (FaMA) is a framework for the automated analysis~\cite{} of feature models. Among the analysis included, we can find ... Nevertheless, FaMA analysis feature models created with other feature modelling tools.

Since the state-of-art feature modelling tools do not support the specification of external constraints involving clonable features, FaMA, to the best of our knowledge, does not incoporate any kind of analysis for validating this kind of constraint.

% Nevertheless, the ideas exposed through this paper might be easily integrated into FaMA, since FaMA claims to be % an extensible framework, and the CSP tool we are using, called Choco, it is already integrated into FaMA.


Moskitt Feature Modeliing (MFM) is a graphical editor for feature models, based on the Mooskkit graphical  supports the graphical edition of feature models, including clonable features. Nevertheless, Moskitt only supports the specification of simple binary constraints between features, more specifically, the specification of \emph{requires}, i.e. \imp{A implies B}, and \emph{excludes} relationships. The semantics of these binary relationships when applied to clonable features is undefined.

TO BE COMPLETED







\section{ConclusionsFuture Work}
\label{sec:summary}

%===============================================================================%
% Section: Summary                                             Date: 04/12/2009 %
% Version: 1.0                                                                  %
% Conference: Caise 2010                                                        %
%===============================================================================%



\bibliographystyle{IEEEtrans}
\bibliography{splc2011}

\end{document}
