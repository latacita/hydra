%%==================================================================%%
%% Author : Tejedo Gonz�lez, Daniel                                 %%
%%          S�nchez Barreiro, Pablo                                 %%
%% Version: 1.0, 18/11/2012                                         %%                   
%% Version: 1.0, 06/02/2013                                         %%                   
%%                                                                  %%
%% Memoria del Proyecto Fin de Carrera                              %%
%% Antecedentes, arquitectura de plugins de eclipse                 %%
%%==================================================================%%

En entorno de desarrollo Eclipse es un ejemplo de arquitectura modular f�cilmente extensible mediante una compleja, pero sencilla al programador, arquitectura de \emph{plug-ins}. Un \emph{plug-in} en Eclipse es un componente que provee un cierto tipo de servicio dentro del contexto del espacio de trabajo de Eclipse. Es decir, una herramienta que se puede integrar en el entorno Eclipse junto con sus otras funcionalidades. Dado que la herramienta \emph{Hydra} fue dise�ada como un \emph{plug-in} para Eclipse, y nuestro editor pretende integrarse tanto en \emph{Hydra} como en \emph{Eclipse}, es necesario conocer y manejar el funcionamiento de la arquitectura de plug-ins de Eclipse.

%%==============================================================================================%%
%% NOTA(Pablo): Esto no se entiende nada
%%==============================================================================================%%
%%
%% En particular, se han utilizado mucho los puntos de extensi�n. Un punto de extensi�n en un
%% plug-in indica la posibilidad de que ese plug-in sea a su vez parte de otro, o que haya 
%% otros plug-ins que sean parte de �l. Esta particularidad permite no s�lo la integraci�n de 
%% nuestro editor con Hydra, sino tambi�n la personalizaci�n de men�s y botones para �l 
%% gracias a la creaci�n de puntos de extensi�n con plug-ins de creaci�n de men�s y barras de
%% herramientas.
%%
%%==============================================================================================%%

%%==============================================================================================%%
%% NOTA(Pablo): Para solucionar
%% - Describir en uno o dos p�rrafos c�mo funciona la arquietctura de plug-ins para Eclipse
%% - Poner un ejemplo de punto de extensi�n, sencillo y concreto, y explicar como funciona 
%%   el punto de extensi�n utilizando algo de c�digo.
%% Si no sabes como escribir esta secci�n, la eliminas directamente, y actualizas la intro 
%% al Cap�tulo de forma conveniente.
%%==============================================================================================%%
