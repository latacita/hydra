%%==================================================================%%
%% Author : Tejedo Gonz�lez, Daniel                                 %%
%%          S�nchez Barreiro, Pablo                                 %%
%% Version: 1.0, 28/11/2012                                         %%                   %%                                                                  %%
%% Memoria del Proyecto Fin de Carrera                              %%
%% Validation Framework, requisitos                                      %%
%%==================================================================%%

La captura de requisitos del proceso de validaci�n pasa por detectar qu� aspectos de las sintaxis concretas que construyamos son susceptibles de convertirla en err�nea, y que adem�s no pueden ser detectados por los mecanismos restrictivos proporcionados por la gram�tica y el metamodelo. Los requisitos encontrados han sido fundamentalmente los siguientes:

- Comprobaci�n de que la direcci�n introducida en el import para cargar el modelo de caracter�sticas al que se aplicar�n las restricciones es correcta. Ser correcta no significa tanto que est� en un formato de direcci�n v�lido (esto se detecta por la gram�tica, salvo quiz�s alguna trampa espec�fica puesta a prop�sito), sino que la direcci�n especificada contenga un fichero xmi con un modelo de caracter�sticas.

- Comprobaci�n de que las caracter�sticas escritas existan en el modelo importado. L�gicamente, no tendr�a sentido permitir escribir caracter�sticas que no est�n presentes en ese modelo, pues luego a la hora de evaluar las restricciones en las que estuvieran presentes nunca iban a estar seleccionadas. 

- Comprobaci�n de que una caracter�stica parseada como simple realmente lo sea. Cuando en una restricci�n tiene una operaci�n l�gica, se fuerza a que sus operandos sean parseados como caracter�sticas simples, ya que son las �nicas que pueden evaluarse a 1 � 0. En caso de poner una caracter�stica que realmente es m�ltiple en una operaci�n l�gica provocar�a un error en la ejecuci�n, pues al evaluarse podr�a tener un valor mayor que 1. En las operaciones en que se fuerza que las caracter�sticas sean parseadas a m�ltiples (como las de comparaci�n) esto no es un problema, ya que las simples se pueden ver como un caso concreto de las m�ltiples. Por eso, aunque sea parseada a m�ltiple y en realidad sea simple no tiene importancia y la operaci�n seguir� teniendo sentido.

 