%%==================================================================%%
%% Author : Tejedo Gonz�lez, Daniel                                 %%
%%          S�nchez Barreiro, Pablo                                 %%
%% Version: 1.0, 18/11/2012                                         %%                   
%% Version: 2.0, 05/02/2013                                         %%                   
%%                                                                  %%
%% Memoria del Proyecto Fin de Carrera                              %%
%% Conclusiones, conclusiones personales                       %%
%%==================================================================%%

Para el desarrollo de este editor se han utilizado herramientas que son est�ndares \emph{defacto} dentro de la comunidad de modelado, tales como Ecore o EMF. Estas herramientas ofrecen funcionalidades bastante potentes, pero dada su reciente aparici�n, a�n presentan ciertas carencias que han de ser resueltas.

Por ejemplo, el editor gr�fico para metamodelos de Ecore podr�a mejorarse en muchos aspectos.  No estar�a mal poder ser capaz de editar algunos de los par�metros de las metaclases y sus relaciones sin tener que cambiar de ventana. Tambi�n mejorar�a la propia interfaz de colocaci�n de las metaclases, pues colocar las relaciones del modo que uno desea es una tarea mucho m�s ardua de lo que deber�a ser.

En el caso de la herramienta \emph{EMFText}, ni siquiera estamos hablando de una utilidad que est� reconocida como est�ndar \emph{de facto}. Tiene una vida muy corta y apenas se puede decir que est� en fase beta. En muchos momentos del desarrollo nos encontramos con errores que los propios desarrolladores desconoc�an, y tuve que intercambiar e-mails con ellos para poder solucionar ciertos aspectos de nuestra gram�tica.

El mayor freno que sufri� el desarrollo fue debido en la mayor�a de los casos a problemas con las propias aplicaciones. A�n as�, fue muy edificante poder desarrollar un software siguiendo una metodolog�a tan radicalmente distinta de lo que en muchas ocasiones aprend� a lo largo de mis estudios. Creo que la construcci�n de Software Dirigido por Modelos tiene mucho futuro, y en cuanto las herramientas consigan estandarizarse no me cabe duda que ser�n utilizadas de un modo mucho m�s frecuente en el futuro.