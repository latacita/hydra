%%==================================================================%%
%% Author : Tejedo Gonz�lez, Daniel                                 %%
%%          S�nchez Barreiro, Pablo                                 %%
%% Version: 1.0, 18/11/2012                                         %%                   
%% Version: 2.0, 05/02/2013                                         %%                   
%%                                                                  %%
%% Memoria del Proyecto Fin de Carrera                              %%
%% Conclusiones, trabajos futuros                       %%
%%==================================================================%%

Dado que los �rboles de Caracter�sticas son un concepto muy reciente en la historia de la computaci�n, cada poco tiempo surgen nuevas ampliaciones e ideas para potenciar sus virtudes, adem�s de nuevos conceptos adyacentes que permiten complementar su labor. Por eso esta herramienta tiene mucho margen de mejora en el futuro, ya que pronto ser� necesario incorporar muchas de esas mejoras a nuestro editor de restricciones.

Por ejemplo, el editor podr�a mejorarse en el futuro incorporando las siguientes caracter�sticas:

\begin{enumerate}

	\item Incorporar cierta capacidad de an�lisis a los modelos de caracter�sticas, de modo que a trav�s del editor de restricciones podamos indicar que el �rbol que estamos analizando algunos par�metros como un m�ximo de profundidad para sus ramas, o su anchura, etc.

	\item Extender el lenguaje para soportar el novedoso concepto de caracter�sticas con atributos, y adaptar las restricciones posibles a esta novedad.

	\item Extender el lenguaje para permitir especificar restricciones sobre �rboles de caracter�sticas especificados en varios ficheros.

	\item Implementar funcionalidad de \emph{Live Validation}, es decir, que una restricci�n sea evaluada a la vez que est� siendo definida. De este modo se avisar�a lo antes posible de que una restricci�n no est� siendo satisfecha por la configuraci�n.
\end{enumerate}